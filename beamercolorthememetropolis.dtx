% \iffalse meta-comment -------------------------------------------------------
% Copyright 2015 Matthias Vogelgesang and the LaTeX community. A full list of
% contributors can be found at
%
%     https://github.com/matze/mtheme/graphs/contributors
%
% and the original template was based on the HSRM theme by Benjamin Weiss.
%
% This work is licensed under a Creative Commons Attribution-ShareAlike 4.0
% International License (https://creativecommons.org/licenses/by-sa/4.0/).
% ------------------------------------------------------------------------- \fi
% \iffalse
%<driver> \ProvidesFile{beamercolorthememetropolis.dtx}
%<*package>
\NeedsTeXFormat{LaTeX2e}
\ProvidesPackage{beamercolorthememetropolis}
    [2015/06/12 A Modern Beamer Color Theme]
%</package>
%<driver> \documentclass{ltxdoc}
%<driver> \usepackage{beamercolorthememetropolis}
%<driver> \begin{document}
%<driver> \DocInput{beamercolorthememetropolis.dtx}
%<driver> \end{document}
% \fi
% \CheckSum{0}
% \StopEventually{}
% \iffalse
%<*package>
% ------------------------------------------------------------------------- \fi
%
% \subsection{\textsc{metropolis} color theme}
%
% Load required packages.
%    \begin{macrocode}
\RequirePackage{pgfopts}
%    \end{macrocode}
%
%
%
% \subsubsection{Options}
%
% \begin{macro}{block}
% This option controls whether the blocks are filled or transparent.
%    \begin{macrocode}
\pgfkeys{
  /metropolis/color/block/.cd,
    .is choice,
    transparent/.code=\@metropolis@block@transparent,
    fill/.code=\@metropolis@block@fill,
}
%    \end{macrocode}
% \end{macro}
%
% \begin{macro}{colors}
% Defines whether the background shall be dark and the foreground be light or
% vice versa
%    \begin{macrocode}
\pgfkeys{
  /metropolis/color/background/.cd,
    .is choice,
    dark/.code=\@metropolis@colors@dark,
    light/.code=\@metropolis@colors@light,
}
%    \end{macrocode}
% \end{macro}
%
% \begin{macro}{\@metropolis@color@setdefaults}
% Set default values for color theme options.
%    \begin{macrocode}
\newcommand{\@metropolis@color@setdefaults}{
  \pgfkeys{/metropolis/color/.cd,
    background=light,
    block=transparent,
  }
}
%    \end{macrocode}
% \end{macro}
%
%
%
% \subsubsection{Base colors}
%
%    \begin{macrocode}
\definecolor{mDarkBrown}{HTML}{604c38}
\definecolor{mDarkTeal}{HTML}{23373b}
\definecolor{mLightBrown}{HTML}{EB811B}
\definecolor{mLightGreen}{HTML}{14B03D}
%    \end{macrocode}
%
%
%
% \subsubsection{Base styles}
%
% All colors in the \textsc{metropolis} theme are derived from the definitions
% of |normal text|, |alerted text|, and |example text|.
%
%    \begin{macrocode}
\newcommand{\@metropolis@colors@dark}{
  \setbeamercolor{normal text}{%
    fg=black!2,
    bg=mDarkTeal
  }
}
\newcommand{\@metropolis@colors@light}{
  \setbeamercolor{normal text}{%
    fg=mDarkTeal,
    bg=black!2
  }
}
\setbeamercolor{alerted text}{%
  fg=mLightBrown
}
\setbeamercolor{example text}{%
  fg=mLightGreen
}
%    \end{macrocode}
%
%
%
% \subsubsection{Derived colors}
%
% The titles and structural elements (e.g. |itemize| bullets) are set in the
% same color as |normal text|. This would ideally done by setting |normal text|
% as a parent style, which we do to set |titlelike|, but this doesn't work for
% |structure| as its foreground is set explicitly in
% |beamercolorthemedefault.sty|.
%
%    \begin{macrocode}
\setbeamercolor{titlelike}{%
  use=normal text,
  parent=normal text
}
\setbeamercolor{structure}{%
  fg=normal text.fg
}
%    \end{macrocode}
%
% The “primary” palette should be used for the most important navigational
% elements, and possibly of other elements. The \textsc{metropolis} theme uses
% it for frame titles and slides.
%
%    \begin{macrocode}
\setbeamercolor{palette primary}{%
  use=normal text,
  fg=normal text.bg,
  bg=normal text.fg
}
\setbeamercolor{frametitle}{%
  use=palette primary,
  parent=palette primary
}
%    \end{macrocode}
%
% The \textsc{metropolis} inner or outer themes optionally display progress
% bars in various locations. Their color is set by |progress bar| but the two
% different kinds can be customized separately. The horizontal rule on the
% title page is also set based on the progress bar color and can be customized
% with |title separator|.
%
%    \begin{macrocode}
\setbeamercolor{progress bar}{%
  use=alerted text,
  fg=alerted text.fg,
  bg=normal text.bg!50!normal text.fg
}
\setbeamercolor{title separator}{
  use=progress bar,
  parent=progress bar
}
\setbeamercolor{progress bar in head/foot}{%
  use=progress bar,
  parent=progress bar
}
\setbeamercolor{progress bar in section page}{
  use=progress bar,
  parent=progress bar
}
%    \end{macrocode}
%
% Blocks
%
%    \begin{macrocode}
\newcommand{\@metropolis@block@transparent}{
  \setbeamercolor{block title}{use=normal text, parent=normal text}
}
\newcommand{\@metropolis@block@fill}{
  \setbeamercolor{block title}{%
    use=normal text,
    fg=normal text.fg,
    bg=normal text.bg!80!fg
  }
}
\setbeamercolor{block title alerted}{%
    use={block title, alerted text},
    bg=block title.bg,
    fg=alerted text.fg
}
\setbeamercolor{block title example}{%
    use={block title, example text},
    bg=block title.bg,
    fg=example text.fg
}
\setbeamercolor{block body alerted}{use=block body, parent=block body}
\setbeamercolor{block body example}{use=block body, parent=block body}
\setbeamercolor{block body}{
  use={block title, normal text},
  bg=block title.bg!50!normal text.bg
}
%    \end{macrocode}
%
% Footnotes
%
%    \begin{macrocode}
\setbeamercolor{footnote}{fg=normal text.fg!90}
\setbeamercolor{footnote mark}{fg=.}
%    \end{macrocode}
%
% Process package options
%
%    \begin{macrocode}
\@metropolis@color@setdefaults
\ProcessPgfPackageOptions{/metropolis/color}
%    \end{macrocode}
%
%    \begin{macrocode}
\mode<all>
%    \end{macrocode}
%
% \iffalse
%</package>
% \fi
% \Finale
\endinput
